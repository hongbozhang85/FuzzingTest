\section{Critical Crash Number in Crash Model}

Given a set of fuzzing parameters, the expected number of crashes triggered
by active seed will be given in the following.

In a crash model with infinte nodes in Markov chain, which means a seed
could potentially trigger infinite unique crashes
The length of fuzzing campaign is $c$ (campaignlength), within each campaign,
the number of trial is $t$ (trialblock). So the duration $d$ (duration)
in fuzzing simulation will be $d = t \times c$. If the worker in a pull is $w$,
the total number of testing in a fuzzing is $m = d \times w$.
In addition, the number of active seeds is $a$ (active).
The initial probability of triggering a crash is $\lambda_0$ (lambdahigh), and
the discount factor is $\gamma$ (exponentialdecay). The the probability of 
triggering the (n+1)th unique crash will be 
\[
		\lambda_n = \lambda_0 \times \gamma^n
\]

Therefore, the expected number of testing in a fuzzing to trigger the nth 
unique crash by an active seed is
\[
		s(n) = \frac{1}{\lambda_0} + \frac{1}{\lambda_1} + \ldots + \frac{1}{\lambda_{n-1}}
			 = \frac{1}{\lambda_0} \times \frac{\gamma^{n} -1}{\gamma^{n} - \gamma^{n-1}}
\]

Consequently, the expected number of testing in a fuzzing to trigger the nth
unique crash by $a$ active seeds will be $S(n) = a \times s(n)$.

Matching $S(n)$ with $d$ to estimate the value of $n$ in a fuzzing campaign.

\[
		\frac{a}{\lambda_0} \times \frac{\gamma^{n+1} -1}{\gamma^{n+1} - \gamma^n} \approx m = t \times c \times w
\]

From above agreba equation, we can solve $n$.

If $c=10^3, t = 10^2, w= 10^3, a = 15, \lambda_0 = 10^{-4}, \gamma = 0.6$, 
$n$ will be $11 \sim 12$.

Consequently, in a fuzzing campaign with above parameters, in average, an
active seed will actually trigger $11 \sim 12$ unique crashes.


